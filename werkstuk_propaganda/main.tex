\documentclass[11pt]{amsart}

\usepackage{tgpagella}
\fontfamily{qpl}
\usepackage[utf8]{inputenc}
\usepackage{amsmath, amsthm, amscd, amsfonts, amssymb, graphicx, color}
\usepackage[bookmarksnumbered, colorlinks, plainpages=false]{hyperref}
\usepackage{bookmark}
\usepackage{csquotes}

\usepackage[dutch]{babel}

\usepackage[style=verbose-ibid,backend=biber]{biblatex}
\addbibresource{zotero.bib}

\hypersetup{
     colorlinks = true,
     citecolor = blue,
     urlcolor = blue,
     linkcolor = blue,
}

\textheight 22.5truecm \textwidth 14.5truecm
\setlength{\oddsidemargin}{0.35in}\setlength{\evensidemargin}{0.35in}

\setlength{\topmargin}{-.5cm}

\begin{document}
\setcounter{page}{1}

\noindent {\small 2324-S1 Themacollege I WGR 103}\hfill     {\small \today} \\
% title, issn/data
{\small Werkstuk over propaganda tijdens de Nederlandse Opstand.}\hfill
{\small } %volume, url

\centerline{}

\centerline{}

\title[Propaganda tijdens de Opstand]{De communicatieve gebruiken van het Habsburgse regime in
     verhouding met de propaganda van de rebellen - tijdens de Opstand,
     1568 \ldots 1648}

\author[J. Vermeltfoort]{Vermeltfoort, Jenny$^1$}

\address{$^{1}$ 3787494, Faculteit Geesteswetenschappen, Leiden
     Universiteit, Leiden, Nederland.}
\email{\textcolor[rgb]{0.00,0.00,0.84}{j.vermeltfoort@umail.leidenuniv.nl}}

\begin{abstract}
     TODO
\end{abstract}
\maketitle

\section{Introductie en toelichting}


> Hoe komt het dat de communicatie strategieen van het Habsburgse regime zo verschilde met dat van de Republiek?


Oranje had voornamelijk een politieke visie op de propaganda, hierdoor werd er niet in gegaan op de zorgen van de
anti-Spaanse katholieken. Farnese gebruikte het groeiende conflict tussen de katholieken en de calvinisten om zijn
propaganda te versterken.

\noindent Antwerpen, 1584, met een veertien maanden durend beleg kwam de stad in handen van de barmhartige en
strategische bekwame landvoogd, Alexander Farnese. Voorafgaand aan het bewind van deze Habsburgse landvoogd waren de
Spaanse Nederlanden in een wanordelijke staat, er bestond een conflict tussen de provincies en met de opkomende
godsdienstvrijheid liep het tussen de katholieke en calvinisten hoog op. Er heerste
onvrede tegenover het Habsburgse regime na het hardvochtige bewind van de Hertog van Alva en de ontoereikende communicatie van Don Juan.
Farnese speelde strategisch in op de zwakte van de Unie van Utrecht met een
aantal succesvolle herovering van verschillende steden, en door middel van pro-regime propaganda - wat uniek was voor
het regime - wist hij het imago van het Habsburgse regime te herstellen. Calvinisten binnen de heroverde steden werden
niet langer vervolgd, op verdraagzame wijze besloot Farnese de calvinisten de ruimte te geven de veroverde gebieden te verlaten.
Hij wist actief betrokkenheid aan te gaan met het volk door in te spelen op de heersende grieven. Dit versterkte niet
enkel het imago van het regime, maar ook dat van hemzelf. Toch sloeg de landvoogd er niet in de traditionele communicatie
methodieken van het regime volledig te verwerpen en wist zijn technieken niet te institutionaliseren.
Met de vervolgde oorlog naar Engeland en Frankrijk werd het volk van de Spaanse-Nederlanden vergeten. Met enkel 2
Na de vernietiging van de Spaanse Armada in 1588 werd de reputatie van de landvoogd aangetast, raakte de 
eens actieve communicatieve benadering van het regime in verval, wat, samen met diverse andere factoren, 
leidde tot de stagnatie van heroveringsinspanningen.



Wat deed het noorden om de katholieken voor zicht te winnen?


Alexander Farnese (1586-1592), de hertog van Parma, en landvoogd van de Spaanse-Nederlanden (1578-1592). Spaanse-Nederlanden, ofwel de loyalisten, worden in dit artikel gespecificeerd als de zuidelijke Lage Landen. De noordelijke Lage Landen, ofwel de gewesten geunificeerd in de Unie van Utrecht (1579), worden benoemd als de rebellen. Zoals hierboven beschreven kwam Farnese in een tijd van wanorde gezeteld op de landvoogdelijke troon van de zuidelijke Lage Landen. 




\newpage

\section{Het autoritair karakter van het zuiden}
\textit{Schets de situatie van het zuiden en de komst van Alexander Farnese. Wat leidde ertoe dat Farnese ervoor koos het
     publiek te gebruiken en waarom was de communicatie niet langer autoritair, zoals dat van Alva. Vergelijk deze twee
     landvoogden en de situatie waar ze zich in bevonden. Ik negeer Don Juan aangezien hij geen grote invloeden heeft
     uitgeoefend.}

\section{De propaganda van Alexander Farnese}
\textit{Beschrijf de inhoud propaganda van Alexander Farnes, gebruik misschien "Not as bad as all that: The strategies and effectiveness of Loyalist propaganda in the early years of Alexander Farnese's governorship"}

\section{De discussie cultuur van het noorden}
\textit{Schets de situatie van het noorden, beschrijf de discussie maatschappij en plaats hierbinnen de propaganda als een
     instrument om politieke acties uit te voeren. Beschrijf te zwakheden van de staat en hoe dit leidde tot een nationaliteit of identiteit van de Nederland. Ontleed hieruit dat propaganda noodzakelijk was om een identiteit te creeeren tussen de verschillende provincies wie geen verleden met elkaar deelde. }

\section{Conclusie}
\textit{Maak een samengevatte vergelijking welke aspecten zijn er hetzelfde? religie standpunten van de staat (romans 13), de zwakheden, }

\newpage\printbibliography{}

\end{document}

%------------------------------------------------------------------------------
% End of journal.tex
%------------------------------------------------------------------------------
