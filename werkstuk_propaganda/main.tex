\documentclass[11pt]{amsart}

\usepackage{tgpagella}
\fontfamily{qpl}
\usepackage[utf8]{inputenc}
\usepackage{amsmath, amsthm, amscd, amsfonts, amssymb, graphicx, color}
\usepackage[bookmarksnumbered, colorlinks, plainpages=false]{hyperref}
\usepackage{bookmark}
\usepackage{csquotes}

\usepackage[dutch]{babel}

\usepackage[style=verbose-ibid,backend=biber]{biblatex}
\addbibresource{zotero.bib}

\hypersetup{
     colorlinks = true,
     citecolor = blue,
     urlcolor = blue,
     linkcolor = blue,
}

\textheight 22.5truecm \textwidth 14.5truecm
\setlength{\oddsidemargin}{0.35in}\setlength{\evensidemargin}{0.35in}

\setlength{\topmargin}{-.5cm}

\begin{document}
\setcounter{page}{1}

\noindent {\small 2324-S1 Themacollege I WGR 103}\hfill     {\small \today} \\
% title, issn/data
{\small Werkstuk over propaganda tijdens de Nederlandse Opstand.}\hfill
{\small } %volume, url

\centerline{}

\centerline{}

\title[Propaganda tijdens de Opstand]{De communicatieve gebruiken van het Habsburgse regime in
     verhouding met de propaganda van de rebellen - tijdens de Opstand,
     1568 \ldots 1648}

\author[J. Vermeltfoort]{Vermeltfoort, Jenny$^1$}

\address{$^{1}$ 3787494, Faculteit Geesteswetenschappen, Leiden
     Universiteit, Leiden, Nederland.}
\email{\textcolor[rgb]{0.00,0.00,0.84}{j.vermeltfoort@umail.leidenuniv.nl}}

\begin{abstract}
     TODO
\end{abstract}
\maketitle

\section{Introductie en toelichting}

\noindent Antwerpen, 1584, met een veertien maanden durend beleg kwam de stad in handen van de barmhartige en
strategische bekwame landvoogd, Alexander Farnese. Voorafgaand aan het bewind van deze Habsburgse landvoogd waren de
Spaanse Nederlanden in een wanordelijke staat, er bestond een conflict tussen de provincies en met de opkomende
godsdienstvrijheid liep het tussen de katholieke en calvinisten hoog op. Er heerste
onvrede tegenover het Habsburgse regime na het hardvochtige bewind van de Hertog van Alva en de ontoereikende communicatie van Don Juan.
Farnese speelde strategisch in op de zwakte van de Unie van Utrecht met een
aantal succesvolle herovering van verschillende steden, en door middel van pro-regime propaganda - wat uniek was voor
het regime - wist hij het imago van het Habsburgse regime te herstellen. Calvinisten binnen de heroverde steden werden
niet langer vervolgd, op verdraagzame wijze besloot Farnese de calvinisten de ruimte te geven de veroverde gebieden te verlaten.
Hij wist actief betrokkenheid aan te gaan met het volk door in te spelen op de heersende grieven. Dit versterkte niet
enkel het imago van het regime, maar ook dat van hemzelf. Toch sloeg de landvoogd er niet in de traditionele communicatie
methodieken van het regime volledig te verwerpen en wist zijn technieken niet te institutionaliseren.
Met de vervolgde oorlog naar Engeland en Frankrijk werd het volk van de Spaanse-Nederlanden vergeten. Met enkel 2
Na de vernietiging van de Spaanse Armada in 1588 werd de reputatie van de landvoogd aangetast, raakte de
eens actieve communicatieve benadering van het regime in verval, wat, samen met diverse andere factoren,
leidde tot de stagnatie van heroveringsinspanningen.

In het noorden van het de Lage Landen had de publieke arena een uitzonderlijk effect op de identiteit van de
Nederlanden, beschreven door Judith Pollman, en Andrew Spicer\autocite{pollmannPublicOpinionChanging2007}. In de bundel
word door middel van een aantal artikelen de publieke arena en het effect van de noordelijke propaganda geschetst.
Andrew Sawyer en zijn artikel \textit{Medium and Message} laat het karakter zien van de noordelijke propaganda, waar
vrijheid en gerechtigheid een hoofdrol spelen, met een aversie naar monarchische macht en waar er werd gepleit voor een
oligarchische bestuur. Henk van Nierop\autocite{vannieropYeShallHear2007} laat zien dat geruchten net zo belangrijk
zijn als de pamfletten, en het artikel \textit{A Provincial News Community in Sixteenth-Century Europe} van Andrew
Pettegree\autocite{pettegreeProvincialNewsCommunity2007} beschrijft het belang van de lokale nieuws gemeenschappen. De
artikelen binnen deze bundel laten het belang zien van de publieke arena voor de vorm van de Lage Landen.

Jasper van der Steen, met artikel \textit{Remembering the Revolt of the Low Countries Historical Canon Formation in the
     Dutch Republic and Habsburg Netherlands}\autocite{vandersteenRememberingRevoltLow2018}, merkt op dat de het nationaal
karakter van het noorden ontstaat door de zwakte van de staat, religieus verdeeld en ontbrekende van een sterk centraal
gezag.

Het boek van Monica Stensland beschrijft de communicatie methodieken van de verschillende Habsburgse landvoogden. Ze
laat zien dat de traditionele publieke relatie van het Regime voornamelijk een autoritair karakter beslaat en dat de
komst van Farnese resulteerde in een succesvolle propaganda campagne\autocite{stenslandHabsburgCommunicationDutch2012}.
Stensland concludeert dat de manier waarop Farnese zijn publieke relaties benaderd radicaal anders is als de
landvoogden die aan hem vooraf zijn gegaan. Dit artikel zal verklaren wat de reden is waarom zijn voorgangers geen
invloed beoefende in het publieke debat. Er zal ingegaan worden op de manier waarop de Hertog van Alva, en in het kort
zijn opvolger Luis de Requesens y Zúñiga, de communicatie richting het volk bewerkstelligen. Het zal blijken dat de
communicatie van Farnese niet enkel het resultaat was van de initiatief van de landvoogd, maar ook als sterke factor
een natuurlijk gevolg van de situatie is.

\newpage

\section*{Het publieke debat}
\noindent  Tijdens de Nederlandse Opstand ontstond er een werkelijke discussie cultuur. Het was van belang om deel te nemen aan de
publieke opinie om invloed uit te oefenen op de bevolking. Wanneer er wordt gedacht aan de Nederlandse discussie cultuur wordt er al snel gedacht aan de anti-Spaanse, of specifiek anti-Alva, propaganda van Willem van Oranje. De wreedzame en het autoritaire beleid van Hertog van Alva leidde tot de uniformiteit van de rebellen. Dit paragraaf zal inzicht geven in hoe het publieke debat er functioneel uitzag en het zal aantonen dat een aanzienlijk persoon door invloed richting kan geven aan hoe het volk een situatie ondervond.

In de essay van Henk van Nierop word duidelijk dat geruchten een sterke basis vormde van het publieke debat. De teksten
van Wouter Jacbobsz, Godevaert van Haecht, en Marcus van Vaernewijck laten blijken hoe de lokale bevolking enkel
ge"informeerd werd door geruchten. [70, public opinion] De beschrijvingen in de dagboeken werden pas geschreven wanneer
gebeurtenis al een langere tijd in het verleden was, dit impliceert dat er een verificatie process genoodzaakt was om
de geruchten te controleren. [71] Deze controle was geen sterk doorgrond process, het blijkt dat de schrijvers de
geruchten vergeleken met andere onafhankelijke bronnen, er werden getuigen gebruikt voor verificatie of ze werden
vergeleken met een legitiem schrijfstuk. [74-75] Het duid hierop dat de schrijvers niet gemakkelijk een gerucht
accepteerde voor wat het was. Hier zou dus geconcludeerd kunnen worden dat propaganda gebaseerd moet zijn op een
verifieerbare gebeurtenis.

Van Gelderen verklaard dat er een ideologie ontwikkelde tijdens de Nederlandse Opstand. Een ideologie waarin de
vrijheden van de Lage Landen door een politiek kader werd beschermt. Dit kader werd gebaseerd op populaire
Soevereiniteit en functioneerde door constitutionele garanties. [Van Gelderen, Political Thought, 30., Andrew Sawyer,
          166] Andrew Sawyer verklaard dat de artiest van de visuele kunst de politieke agenda moesten plaatsen binnen een
context en waar ze vervolgens de associatie legde aan de Lage Landen. Deze presentaties waren echter geen ge"isoleerde
representaties van de staat, vaak werd deze met de context vermengd. De staat was dus niet gepersonifieerd, maar werd
geplaatst binnen een gebeurtenis. [167] De politieke afbeeldingen bevestigde een weerwil richting dynastie,
hiërarchische macht uitgedrukt in monarchie en bemoedigde de onderhandelend, verspreiden en oligarchische
machtsstructuur. Het afbeelden van een andere manier van besturen was belangrijker dan het afbeelden van weerwil. [187]

In de Nederlandse Republiek in de zeventiende eeuw, die een overvloed aan kunstwerken produceerde, waren de burgers
zeer gevoelig voor visuele kunst, die niet alleen in schilderijen maar ook in andere media, zoals glas-in-loodramen,
werd gebruikt. Beeldtaal werd ook politiek gebruikt, en een van de beste uitdrukkingen van Nederlands politiek denken
over vrijheid van geweten was te vinden in een gebrandschilderd glasraam in de Janskerk in Gouda. Deze visuele
elementen hielpen bij het creëren van een dissidente politieke cultuur. Prenten en gravures werden eveneens gebruikt om
politieke boodschappen over te brengen, vaak in historische of allegorische taferelen. Beeldmateriaal was krachtig om
complexe politieke situaties en relaties uit te drukken. De Republiek gebruikte deze beeldgeletterdheid niet alleen om
een dissidente politieke cultuur te vormen, maar ook in de praktische politiek en besluitvorming in een tijd van oorlog
en dreiging van buitenaf. [163, Sawyer]

In deze centra van handel en handel was informatie zeer gewild om praktische redenen, zoals het waarborgen van de
veiligheid van wegen en het onderhouden van zakelijke relaties. Maar de honger naar nieuws ging verder dan deze zorgen.
Het werd erkend door de machthebbers dat het vormgeven van de publieke opinie essentieel was, wat leidde tot de
ontwikkeling van een opkomende publieke opinie in deze samenlevingen. Ten tweede hebben wetenschappers hun aandacht
gericht op de rol van drukwerk bij het vormgeven van politiek bewuste en coöperatieve publieken. Drukwerk speelde een
cruciale rol bij het beïnvloeden van de publieke opinie, en overheden gebruikten het niet alleen om officiële bevelen
te publiceren, maar ook om discussies en interpretaties van actuele gebeurtenissen vorm te geven. Wetten, voorschriften
en verordeningen werden traditioneel mondeling overgebracht, en dit proces zette zich voort in de zestiende eeuw. In
Frankrijk kreeg de mondelinge publicatie van officiële edicten een steeds meer geritualiseerd karakter. In Parijs
werden edicten eerst geregistreerd in het Parlement van Parijs en vervolgens afgekondigd door de officiële omroeper op
aangewezen kruispunten en openbare plaatsen. De omroeper werd vergezeld door de koninklijke trompettist; voor edicten
van bijzonder belang werden meerdere trompettisten voorgeschreven. [35, Pettegree]

De propaganda van de rebellen had als doelgroep de rebellen, niet sceptici. In de pamfletten van de loyalisten waren
gedoeld op het verzoende volk.

\section*{Traditie}
\noindent Karel V had uitgebreide censuurwetgeving uitgegeven in 1544, hierin werd de doodstraf voorgeschreven aan iedereen die werd betrapt op het drukken van werken die propagandistische tekenen bevatten. Bovendien werden de drukkers voorzien van drukkersprivilege, drukken zonder deze privileges was strafbaar. Filips II vaardigde deze wet opnieuw uit in 1556 met nieuwe vereisten, zoals het censuur op gebrandschilderde ramen.[36] Paul Kleber Monod merkt op dat alle vroegmoderne vorsten zichzelf als de door God gegeven autoritaire macht bezitten. Daarom zou het lanceren van een argumentatieve campagne als reactie op rebelse pamfletten kunnen suggereren dat de aantijgingen van de rebellen als legitiem werden erkend.[35] Dit censuur laat zien dat de Habsburgers zich bewust waren van de dreigende effecten van het publieke debat en daarmee de spelers hierbinnen.

De vervolging die Alva in gang zette en de angst die hij opwekte, vormden een beoogd en integraal onderdeel van de
'good cop, bad cop'-strategie. Exemplarische straffen en de verwachte afschrikkende effecten werden gezien als de
primaire middelen om verdere uitbarstingen van beeldstormen en rebellie te voorkomen. Het ongekend hoge aantal
arrestaties en de concentratie van executies maakten straf tot een belangrijk middel voor publieke communicatie. Door
het straffen van mensen die werden bestempeld als rebellen en ketters, gaf het regime een krachtige aanklacht tegen
alle vormen van religieuze en politieke ontrouw, hoe klein ook. Het was niet ongebruikelijk om openbare straffen op
deze manier te gebruiken. Behalve in bijzondere gevallen waren executies theaterevenementen die in het openbaar
plaatsvonden voor een groot publiek. [29]

Openbare vieringen en festivalleen werden door Alva bevorderd. Na de slag bij Jemmingen in juli 1568, waarbij de
troepen van Louis van Nassau vrijwel volledig waren vernietigd, schreef Alva aan de Raad van State om hen op de hoogte
te stellen van de overwinning. Hij benadrukte de grote verliezen die de vijand had geleden, ongeveer zeven duizend
tegenover slechts zes of zeven van zijn eigen mannen. Hij beschouwde de overwinning als een genade van God en stelde
voor dat dit breed bekend zou worden gemaakt, zodat passende dankbetuigingen konden worden gehouden. Op die manier zou
God wellicht blijven helpen bij het "uitroeien" van degenen die de openbare orde verstoorden. Het regime zocht actief
naar de breedst mogelijke verspreiding van de katholieke overwinning. Dit was niets nieuws: al eerder had Karel V
minstens 63 Nederlandstalige pamfletten gesponsord, en minstens twaalf in het Frans, tussen 1520 en zijn abdicatie in
1555. En de strategie werkte: het dagboek van Van Campene maakt duidelijk dat hij informatie over de strijd uit
verschillende bronnen haalde, en zowel zijn als andere schijnbare acceptatie van het idee dat goddelijke steun
instrumenteel was geweest bij het veiligstellen van de overwinning, suggereert dat in ieder geval een deel van het
luister- en leespubliek de boodschap die het regime probeerde te verspreiden, ter harte nam.

Het doel van Requesen was om de het imago van Alva los te maken van het Habsburgse regime. Verlangend naar verzoening,
werd er in 1574 een algemeen pardon afgegeven waarin de rebellen, voornamelijke Willem van Oranje, vergeven werden. Dit
had echter niet het juiste resultaat, door een pardon af te geven word er geimpliceerd dat de schuld bij de rebellen
lag. Hierdoor wist Requesen niet volledig afstand te doen van Alva, het laat zien dat hij zelfs de kant van Alva
verkoos over de kant van het getraumatiseerd volk. [64] Het pardon werd zelfs door de rebellen later gebruikt om te
laten zien dat het bewind van Requesen niet anders is als dat van Alva. [65] Dit is een sterk bewijsstuk van de
onwilligheid van Requesen om afstand te doen aan traditionele methodes van communiceren. Het Habsburgse regime werd in
zijn ogen nog steeds als autoritair gezien. Dit is verder in lijn met de bedwingende autoritaire macht beschreven door
Menod.

\section*{Alexander Farnese}
\noindent Het beleid van Alexander Farnese kan in tegenstelling tot het bewind van de Hertog van Alva gekarakteriseerd worden als zorgvuldig en verzoenend. Zoals al beschreven in de inleiding besloot Alexander Farnese na de overwinning van Antwerpen niet de ketters te vervolgen, maar ze een alternatief te bieden. Farnese transformeerde deze Reconciliatie in zijn persoonlijke propaganda. [20, soen]

De aanval op de katholieke in de Generaliteit in 1578 resulteerde in Belgische pamphleteering, waarin er werd verlangd
dat de oppositie het beleid van de Pacificatie van Gent zouden volgen, hierin stond namelijk dat enige aanval op de
katholieke verboden was. De vastberadenheid van de Waalse provincies om de Pacificatie te handhaven ter verdediging van
hun verzoening is begrijpelijk. [89, stenlands] Farnese werd onmiddellijk tijdens de ingang van dan landvoogdij
genoodzaakt de termen van de Pacificatie van Gent te volgen. De Belgische provincies waren onbuigzaam en vereisten dat
de Spaanse troepen uit de provincies verdwenen. Met het oog op zijn Reconciliatie kon Farnese niet anders dan hiermee
akkoord gaan.

Opvallend is dat Alexander Farnese er wel voor koos om de rebelse aantijgingen als legitiem te beschouwen, aangezien
direct reactie op idealen van de rebellen. Dit gaat in tegen wat een autoritaire macht als de Habsburgers zouden doen.
Een concreet voorbeeld hiervan is anti-Oranje propaganda van Farnese en cardinal Granvelle. [93, stenland] Was dit
slechts een verandering van tacktiek, of was dit een natuurlijk gevolg van de huidige situatie?

\section*{Conclusie}
Communicatie was niet geïnstitutionaliseerd, hierdoor was er een verband tussen het karakter van de landvoogd en dat
van de communicatie. De Spaanse Lage Landen hadden geen verzwakte staat vergeleken met dat van de Republiek. De
deelname aan de publieke opinie kwam pas nadat de ontevredenheid van het volk aanzienlijk was gegroeid.

\newpage\printbibliography{}

\end{document}

%------------------------------------------------------------------------------
% End of journal.tex
%------------------------------------------------------------------------------
