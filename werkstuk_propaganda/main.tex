\documentclass[12pt]{amsart}

\usepackage{times}
\usepackage[utf8]{inputenc}
\usepackage{amsmath, amsthm, amscd, amsfonts, amssymb, graphicx, color}
\usepackage[bookmarksnumbered, colorlinks, plainpages=false]{hyperref}
\usepackage{bookmark}
\usepackage{csquotes}

\usepackage[dutch]{babel}

\usepackage[backend=biber]{biblatex}
\addbibresource{zotero.bib}


\hypersetup{
     colorlinks   = true,
     citecolor    = blue
}


\textheight 22.5truecm \textwidth 14.5truecm
\setlength{\oddsidemargin}{0.35in}\setlength{\evensidemargin}{0.35in}

\setlength{\topmargin}{-.5cm}

\newtheorem{theorem}{Theorem}[section]
\newtheorem{lemma}[theorem]{Lemma}
\newtheorem{proposition}[theorem]{Proposition}
\newtheorem{corollary}[theorem]{Corollary}
\theoremstyle{definition}
\newtheorem{definition}[theorem]{Definition}
\newtheorem{example}[theorem]{Example}
\newtheorem{exercise}[theorem]{Exercise}
\newtheorem{conclusion}[theorem]{Conclusion}
\newtheorem{conjecture}[theorem]{Conjecture}
\newtheorem{criterion}[theorem]{Criterion}
\newtheorem{summary}[theorem]{Summary}
\newtheorem{axiom}[theorem]{Axiom}
\newtheorem{problem}[theorem]{Problem}
\theoremstyle{remark}
\newtheorem{remark}[theorem]{Remark}
\numberwithin{equation}{section}

\begin{document}
\setcounter{page}{1}


\noindent {\small 2324-S1 Themacollege I WGR 103}\hfill     {\small \today} \\ % title, issn/data
{\small Werkstuk over propaganda tijdens de Nederlandse Opstand.}\hfill  {\small } %volume, url

\centerline{}

\centerline{}

\title[Short Title]{De communicatieve gebruiken van het Habsburgse regime in verhouding met de propaganda van de rebellen - tijdens de Opstand, 1568\ldots1648}

\author[F. Author]{Vermeltfoort, Jenny$^1$}

\address{$^{1}$ 3787494, Student Faculteit Geesteswetenschappen, Leiden Universiteit, Leiden, Nederland.}
\email{\textcolor[rgb]{0.00,0.00,0.84}{j.vermeltfoort@umail.leidenuniv.nl}}


%\dedicatory{This paper is dedicated to Professor ABCD}
%\subjclass[2020]{Primary 46L55; Secondary 44B20.}
%\keywords{Analysis, PDEs, algebra, number theory, applied mathematics}
%\date{Received: xxxxxx; Revised: yyyyyy; Accepted: zzzzzz.
%\newline \indent $^{*}$ Corresponding author}

\begin{abstract}
\end{abstract} \maketitle

\section{Introductie en toelichting}

% Hoe manifesteerde de propaganda van het noorden
% Hoe manifesteerde de communicatie van het regime, wat was de reactie op het noorden, 
% 

% de verhouding op het gebied van sociale verschillen tussen het noorden en zuid, denk aan religie, zwakheden van het noorden, gebruik Jasper van de Steen. Propaganda werd een noodzaak, mogelijk een gevolg van de publieke arena, bookshop van de wereld (Andrew Pettegree)

% De reactie en effect van het regime op de propaganda, en hun ondermijning van de nederlandse dissaffectie, Paul Regan, 
% discern dismay about abla's violence, Gustaaf Janssens
% local officals obstructed laws against heresy, Juliaan Wolter's

% kort vertellen over de communcatie methodes van het regime waren aan de hand van monica stenlands, tradionele communcatie, romans 13, etc
% vertellen over de latere propaganda van het regime welke met Don Jou, Farnesse en aartsbishoppen toch manifesteerde, monica stenlands

% Dit artikel zal omschrijven wat was de verhouding tussen de Habsburgse communicatie en de politieke propaganda van de Rebellen?

% Zoals eerder beschreven is de reatie van het regime ingeperkt gebleven en zorgde het niet voor het verwachte effect, vandaar de vraag waarom er vast werd gehouden aan de tradionele methoden en waarom later de propaganda toch manifesteerde. Omschrijven hoe de latere propaganda van het regime veranderd was vergelekene met dat van de originele communicatie methoes, onderdrukking. Waar kwam dit uit voort? Leerde ze van het noorden?

% wat was het effect van de habsburgse propaganda op de noordelijke propaganda?

% conclusie: het noorden maakten sterk gebruik van propaganda tijdens de. De propaganda kwam tot bloei door de verdrukking van het zuiden en de intellectuele infrastructuur in het noorden. Het zuiden onderschatte de ontevredenheid van het volk tijdens de destructieve periode van Alba. 

% religieus verschil
% infrastructuur verschil
% situatie verandering na alva.
\noindent 

\autocite{pollmannPublicOpinionChanging2007}


\printbibliography{}

\end{document}

%------------------------------------------------------------------------------
% End of journal.tex
%------------------------------------------------------------------------------
