\documentclass[11pt]{amsart}

\usepackage{tgpagella}
\fontfamily{qpl}
\usepackage[utf8]{inputenc}
\usepackage{amsmath, amsthm, amscd, amsfonts, amssymb, graphicx, color}
\usepackage[bookmarksnumbered, colorlinks, plainpages=false]{hyperref}
\usepackage{bookmark}
\usepackage{csquotes}

\usepackage[dutch]{babel}

\usepackage[style=verbose-ibid,backend=biber]{biblatex}
\addbibresource{zotero.bib}

\hypersetup{
     colorlinks = true,
     citecolor = blue,
     urlcolor = blue,
     linkcolor = blue,
}

\textheight 22.5truecm \textwidth 14.5truecm
\setlength{\oddsidemargin}{0.35in}\setlength{\evensidemargin}{0.35in}

\setlength{\topmargin}{-.5cm}

\begin{document}
\setcounter{page}{1}

\noindent {\small 2324-S1 Themacollege I WGR 103}\hfill     {\small \today} \\
% title, issn/data
{\small Werkstuk over propaganda tijdens de Nederlandse Opstand.}\hfill
{\small } %volume, url

\centerline{}

\centerline{}

\title[Propaganda tijdens de Opstand]{De communicatieve gebruiken van het Habsburgse regime in
     verhouding met de propaganda van de rebellen - tijdens de Opstand,
     1568 \ldots 1648}

\author[J. Vermeltfoort]{Vermeltfoort, Jenny$^1$}

\address{$^{1}$ 3787494, Faculteit Geesteswetenschappen, Leiden
     Universiteit, Leiden, Nederland.}
\email{\textcolor[rgb]{0.00,0.00,0.84}{j.vermeltfoort@umail.leidenuniv.nl}}

%\dedicatory{This paper is dedicated to Professor ABCD}
%\subjclass[2020]{Primary 46L55; Secondary 44B20.}
%\keywords{Analysis, PDEs, algebra, number theory, applied mathematics}
%\date{Received: xxxxxx; Revised: yyyyyy; Accepted: zzzzzz.
%\newline \indent $^{*}$ Corresponding author}

\begin{abstract}
     TODO
\end{abstract}
\maketitle

\section{Introductie en toelichting}

% Hoe manifesteerde de propaganda van het noorden
% Hoe manifesteerde de communicatie van het regime, wat was de reactie op het noorden, 
% 

% de verhouding op het gebied van sociale verschillen tussen het noorden en zuid, denk aan religie, zwakheden van het noorden, gebruik Jasper van de Steen. Propaganda werd een noodzaak, mogelijk een gevolg van de publieke arena, bookshop van de wereld (Andrew Pettegree)

% De reactie en effect van het regime op de propaganda, en hun ondermijning van de nederlandse dissaffectie, Paul Regan, 
% discern dismay about abla's violence, Gustaaf Janssens
% local officals obstructed laws against heresy, Juliaan Wolter's

% kort vertellen over de communcatie methodes van het regime waren aan de hand van monica stenlands, traditionele communcatie, romans 13, etc
% vertellen over de latere propaganda van het regime welke met Don Juan, Farnesse en aartsbishoppen toch manifesteerde, monica stenlands

% Dit artikel zal omschrijven wat was de verhouding tussen de Habsburgse communicatie en de politieke propaganda van de Rebellen?

% Zoals eerder beschreven is de reatie van het regime ingeperkt gebleven en zorgde het niet voor het verwachte effect, vandaar de vraag waarom er vast werd gehouden aan de traditionele methoden en waarom later de propaganda toch manifesteerde. Omschrijven hoe de latere propaganda van het regime veranderd was vergelekene met dat van de originele communicatie methoes, onderdrukking. Waar kwam dit uit voort? Leerde ze van het noorden?

% wat was het effect van de habsburgse propaganda op de noordelijke propaganda?

% conclusie: het noorden maakten sterk gebruik van propaganda tijdens de. De propaganda kwam tot bloei door de verdrukking van het zuiden en de intellectuele infrastructuur in het noorden. Het zuiden onderschatte de ontevredenheid van het volk tijdens de destructieve periode van Alba. 

% religieus verschil
% infrastructuur verschil
% situatie verandering na alva.

Waarom maakten de rebellen meer gebruik van politieke propaganda dan Filips II?

> Wat verklaard dat Farnese de anti-Spaanse katholieken in de Generaliteitslanden wist te bekeren tot pro-Habsburgs denken?
> Hoe won Farnese de propaganda strijd met de rebellen, waar de anti-Spaanse katholieken tot pro-Habsburgs denken werden bekeerd?
> Hoe wist Farnese tijdens de propagandastrijd met de rebellen de anti-Spaanse katholieken te overtuigen om pro-Habsburgse overtuigingen te omarmen?

Hoe slaagde Alexander Farnese erin om de communicatiestrategie van het Habsburgse regime effectief af te stemmen op de situatie? 


>> Waarom waren er anti-Spaanse katholieken in de Generaliteitslanden?
>> Hoe manifesteerde de propagandastrijd zich tijdens het bewind van Farnese?
>> Hoe zag de publieke arena van de Generaliteitslanden er uit?
>> Waar speelde Farnese's propaganda op in?
>> De reconsille was de enigste optie voor Farnese, hij kon namelijk de calvinisten niet in het land laten zonder dat hier opstanden uit ontstaan. Religievrede was uitgesloten door het katholieke karakter van de Spanjaarden. Het vervolgen van de calvinisten zorgde voor meer disorde en afgunst onder de katholieken, wat blijkt uit Alva's bewind. Ook is hier blijvende belegering noodzakelijk om de opstanden te onderdrukken, waar het volk de oorlogen juist bue waren. Ze zochten naar vrede. Dus uiteindelijk had de vervolging geleid tot een burgeroorlog.
>> Het volk van de Generaliteitslanden waren in motief conflict met de noordelijke gewesten. Aangezien de Generaliteitslanden voornamlijke de dupe zijn geweest van de oorlog wilden zij een einde aan de oorlog en vrede sluiten. Aangezien de rebellen de Generaliteitslanden wilden verlossen leidde dit tot doorzetting van de oorlog, wat in direct conflict was met de motieven van de generaliteitslanden. 


Oranje had voornamelijk een politieke visie op de propaganda, hierdoor werd er niet in gegaan op de zorgen van de
anti-Spaanse katholieken. Farnese gebruikte het groeiende conflict tussen de katholieken en de calvinisten om zijn
propaganda te versterken.

\noindent Antwerpen, 1584, met een veertien maanden durend beleg kwam de stad in handen van de barmhartige en
strategische bekwame landvoogd, Alexander Farnese. Voorafgaand aan het bewind van deze Habsburgse landvoogd waren de
Spaanse Nederlanden in een wanordelijke staat, er bestond een conflict tussen de provincies en met de opkomende
godsdienstvrijheid liep het tussen de katholieke en calvinisten hoog op. Er heerste
onvrede tegenover het Habsburgse regime na het hardvochtige bewind van de Hertog van Alva en de ontoereikende communicatie van Don Juan.
Farnese speelde strategisch in op de zwakte van de Unie van Utrecht met een
aantal succesvolle herovering van verschillende steden, en door middel van pro-regime propaganda - wat uniek was voor
het regime - wist hij het imago van het Habsburgse regime te herstellen. Calvinisten binnen de heroverde steden werden
niet langer vervolgd, op verdraagzame wijze besloot Farnese de calvinisten de ruimte te geven de veroverde gebieden te verlaten.
Hij wist actief betrokkenheid aan te gaan met het volk door in te spelen op de heersende grieven. Dit versterkte niet
enkel het imago van het regime, maar ook dat van hemzelf. Toch sloeg de landvoogd er niet in de traditionele communicatie
methodieken van het regime volledig te verwerpen en wist zijn technieken niet te institutionaliseren.
Met de vervolgde oorlog naar Engeland en Frankrijk werd het volk van de Spaanse-Nederlanden vergeten. Met enkel 2
Na de vernietiging van de Spaanse Armada in 1588 werd de reputatie van de landvoogd aangetast, raakte de 
eens actieve communicatieve benadering van het regime in verval, wat, samen met diverse andere factoren, 
leidde tot de stagnatie van heroveringsinspanningen.



Wat deed het noorden om de katholieken voor zicht te winnen?


Alexander Farnese (1586-1592), de hertog van Parma, en landvoogd van de Spaanse-Nederlanden (1578-1592). Spaanse-Nederlanden, ofwel de loyalisten, worden in dit artikel gespecificeerd als de zuidelijke Lage Landen. De noordelijke Lage Landen, ofwel de gewesten geunificeerd in de Unie van Utrecht (1579), worden benoemd als de rebellen. Zoals hierboven beschreven kwam Farnese in een tijd van wanorde gezeteld op de landvoogdelijke troon van de zuidelijke Lage Landen. 




\newpage

\section{Het autoritair karakter van het zuiden}
\textit{Schets de situatie van het zuiden en de komst van Alexander Farnese. Wat leidde ertoe dat Farnese ervoor koos het
     publiek te gebruiken en waarom was de communicatie niet langer autoritair, zoals dat van Alva. Vergelijk deze twee
     landvoogden en de situatie waar ze zich in bevonden. Ik negeer Don Juan aangezien hij geen grote invloeden heeft
     uitgeoefend.}

\section{De propaganda van Alexander Farnese}
\textit{Beschrijf de inhoud propaganda van Alexander Farnes, gebruik misschien "Not as bad as all that: The strategies and effectiveness of Loyalist propaganda in the early years of Alexander Farnese's governorship"}

\section{De discussie cultuur van het noorden}
\textit{Schets de situatie van het noorden, beschrijf de discussie maatschappij en plaats hierbinnen de propaganda als een
     instrument om politieke acties uit te voeren. Beschrijf te zwakheden van de staat en hoe dit leidde tot een nationaliteit of identiteit van de Nederland. Ontleed hieruit dat propaganda noodzakelijk was om een identiteit te creeeren tussen de verschillende provincies wie geen verleden met elkaar deelde. }

\section{Conclusie}
\textit{Maak een samengevatte vergelijking welke aspecten zijn er hetzelfde? religie standpunten van de staat (romans 13), de zwakheden, }

\newpage\printbibliography{}

\end{document}

%------------------------------------------------------------------------------
% End of journal.tex
%------------------------------------------------------------------------------
