\documentclass[11pt]{amsart}

\usepackage{tgpagella}
\fontfamily{qpl}
\usepackage[utf8]{inputenc}
\usepackage{amsmath, amsthm, amscd, amsfonts, amssymb, graphicx, color}
\usepackage[bookmarksnumbered, colorlinks, plainpages=false]{hyperref}
\usepackage{bookmark}
\usepackage{csquotes}

\usepackage[dutch]{babel}

\usepackage[style=verbose-ibid,backend=biber]{biblatex}
\addbibresource{zotero.bib}

\hypersetup{
     colorlinks = true,
     citecolor = blue,
     urlcolor = blue,
     linkcolor = blue,
}

\textheight 22.5truecm \textwidth 14.5truecm
\setlength{\oddsidemargin}{0.35in}\setlength{\evensidemargin}{0.35in}

\setlength{\topmargin}{-.5cm}

\begin{document}
\setcounter{page}{1}

\noindent {\small 2324-S1 Themacollege I WGR 103}\hfill     {\small \today} \\
% title, issn/data
{\small Werkstuk over propaganda tijdens de Nederlandse Opstand.}\hfill
{\small } %volume, url

\centerline{}

\centerline{}

\title[Propaganda tijdens de Opstand]{De communicatieve gebruiken van het Habsburgse regime in
     verhouding met de propaganda van de rebellen - tijdens de Opstand,
     1568 \ldots 1648}

\author[J. Vermeltfoort]{Vermeltfoort, Jenny$^1$}

\address{$^{1}$ 3787494, Faculteit Geesteswetenschappen, Leiden
     Universiteit, Leiden, Nederland.}
\email{\textcolor[rgb]{0.00,0.00,0.84}{j.vermeltfoort@umail.leidenuniv.nl}}

%\dedicatory{This paper is dedicated to Professor ABCD}
%\subjclass[2020]{Primary 46L55; Secondary 44B20.}
%\keywords{Analysis, PDEs, algebra, number theory, applied mathematics}
%\date{Received: xxxxxx; Revised: yyyyyy; Accepted: zzzzzz.
%\newline \indent $^{*}$ Corresponding author}

\begin{abstract}
     TODO
\end{abstract}
\maketitle

\section{Introductie en toelichting}

% Hoe manifesteerde de propaganda van het noorden
% Hoe manifesteerde de communicatie van het regime, wat was de reactie op het noorden, 
% 

% de verhouding op het gebied van sociale verschillen tussen het noorden en zuid, denk aan religie, zwakheden van het noorden, gebruik Jasper van de Steen. Propaganda werd een noodzaak, mogelijk een gevolg van de publieke arena, bookshop van de wereld (Andrew Pettegree)

% De reactie en effect van het regime op de propaganda, en hun ondermijning van de nederlandse dissaffectie, Paul Regan, 
% discern dismay about abla's violence, Gustaaf Janssens
% local officals obstructed laws against heresy, Juliaan Wolter's

% kort vertellen over de communcatie methodes van het regime waren aan de hand van monica stenlands, traditionele communcatie, romans 13, etc
% vertellen over de latere propaganda van het regime welke met Don Juan, Farnesse en aartsbishoppen toch manifesteerde, monica stenlands

% Dit artikel zal omschrijven wat was de verhouding tussen de Habsburgse communicatie en de politieke propaganda van de Rebellen?

% Zoals eerder beschreven is de reatie van het regime ingeperkt gebleven en zorgde het niet voor het verwachte effect, vandaar de vraag waarom er vast werd gehouden aan de traditionele methoden en waarom later de propaganda toch manifesteerde. Omschrijven hoe de latere propaganda van het regime veranderd was vergelekene met dat van de originele communicatie methoes, onderdrukking. Waar kwam dit uit voort? Leerde ze van het noorden?

% wat was het effect van de habsburgse propaganda op de noordelijke propaganda?

% conclusie: het noorden maakten sterk gebruik van propaganda tijdens de. De propaganda kwam tot bloei door de verdrukking van het zuiden en de intellectuele infrastructuur in het noorden. Het zuiden onderschatte de ontevredenheid van het volk tijdens de destructieve periode van Alba. 

% religieus verschil
% infrastructuur verschil
% situatie verandering na alva.

\noindent Antwerpen, 1584, met een veertien maanden durend beleg kwam de stad in handen van de barmhartige en
strategische bekwame landvoogd, Alexander Farnese. Voorafgaand aan het bewind van deze Habsburgse landvoogd waren de
Generaliteitslanden in een wanordelijke staat, er bestond een agitatie tussen de provincies en met de opkomende
godsdienstvrijheid liep het conflict tussen de katholieke en calvinisten hoog op. Er heerste
onvrede tegenover het Habsburgse regime na het hardvochtige bewind van de Hertog van Alva en de ontoereikende communicatie van Don Juan.
Farnese speelde strategisch in op de zwakte van de Generaliteitslanden met een
aantal succesvolle herovering van verschillende steden, en door middel van pro-regime propaganda - wat uniek was voor
het regime - wist hij het imago van het Habsburgse regime te herstellen. Calvinisten binnen de heroverde steden werden
niet langer vervolgd, op verdraagzame wijze besloot Farnese de calvinisten de ruimte te geven het land te verlaten.
Hij wist actief betrokkenheid aan te gaan met het volk door in te spelen op de heersende grieven. Dit versterkte niet
enkel het imago van het regime, maar ook dat van hemzelf. Toch sloeg de landvoogd er niet in de traditionele communicatie
methodieken van het regime volledig te verwerpen en wist zijn technieken niet te institutionaliseren.
Na de verwoesting van de Spaanse Armada in 1588 werd de reputatie van de landvoogd geschaad, wat ertoe leidde dat de
communicatie wegviel, wat met verschillende andere factoren ertoe leidde dat de Generaliteitslanden verloren gingen.

De manier waarop Farnese de weg door zijn publieke relaties trachtte laat de kracht zien van een pakkende communicatie
strategie. Dit wordt gereflecteerd in het boek van Monica Stensland, waar Monica de communicatie methodieken van de
verschillende Habsburgse landvoogden beschrijft. Ze laat zien dat de traditionele publieke relatie van het Regime
voornamelijk een autoritair karakter beslaat en de komst van Farnesede resulteerde in een succesvolle propaganda
campagne\autocite{stenslandHabsburgCommunicationDutch2012}.

In het noorden van het de Lage Landen had de publieke arena een uitzonderlijk effect op de identiteit van de
Nederlanden, beschreven door Judith Pollman, en Andrew Spicer\autocite{pollmannPublicOpinionChanging2007}. In de bundel
word door middel van een aantal artikelen de publieke arena en het effect van de noordelijke propaganda geschetst.
Andrew Sawyer en zijn artikel \textit{Medium and Message} laat het karakter zien van de noordelijke propaganda, waar
vrijheid en gerechtigheid een hoofdrol spelen, met een aversie naar monarchische macht en waar er werd gepleit voor een
oligarchische bestuur. Henk van Nierop\autocite{vannieropYeShallHear2007} laat zien dat geruchten net zo belangrijk
zijn als de pamfletten, en het artikel \textit{A Provincial News Community in Sixteenth-Century Europe} van Andrew
Pettegree\autocite{pettegreeProvincialNewsCommunity2007} beschrijft het belang van de lokale nieuws gemeenschappen. De
artikelen binnen deze bundel laten het belang zien van de publieke arena voor de vorm van de Lage Landen.

Jasper van der Steen, met artikel \textit{Remembering the Revolt of the Low Countries Historical Canon Formation in the
     Dutch Republic and Habsburg Netherlands}\autocite{vandersteenRememberingRevoltLow2018}, merkt op dat de het nationaal
karakter van het noorden ontstaat door de zwakte van de staat, religieus verdeeld en ontbrekende van een sterk centraal
gezag.

Communicatief is opmerkelijk dat het autoritaire karakter van het Habsburgse regime met de komst van Farnese uitgroeid
tot een

Het is opmerkelijk dat de communicatie van het Habsburgse regime een radicale verandering onderging met de komst van
Farnese. Monica Stenland verklaard dat Farnese de mogelijkheid had

Het is opmerkelijk dat de opkomst van propaganda voortkwam uit tijden van staatsonzekerheid - bij de rebellen, wanneer
zij op het punt stonden privileges te verliezen - tijdens het Habsburgse regime wanneer zij de Generaliteitslanden
verloren. Kan er door een link te leggen tussen de communicatie methodieken van het Habsburgse Regime en de Republiek
een karakteristiek gekoppeld worden aan de 16de-eeuwse propaganda? Was het zo dat de propaganda een facade was voor de
zwakheden van de staten en is de opkomst van propaganda een natuurlijke reactie op de situatie - waar in het Habsburgse
Regime de komst van propaganda enkel gekoppeld wordt aan Farnese door Monica Stensland - in de Republiek het gevolg van
de discussie maatschappij, waar invloed enkel beoefend kan worden binnen de publieke arena, zoals beschreven door
Judith Pollman - of is het ontstaan van propaganda be"invloed door de zwakheden van de noordelijke staat welke Jasper
van der Steen beschrijft? Dit artikel zal een overzicht over de opkomst van propaganda scheppen, en een vergelijking -
gekeken naar motieven, inhoud, en effect - difini"eren van de propaganda van Alexander Farnese en de Republiek in het
algemeen.

Wat Monica Steunlands concludeert is dat de communicatie van het Regime tijdens het landvoogdij van Farnese terug valt
op traditionele methodieken komt voortvloeit uit het ontbreken van militaire successen. Aan deze conclusie ontgaat
echter dat propaganda van het Regime uitgebreider was dan enkel militaire acties. Een deel van de propaganda - de calvinisme vs katholieke dreiging, 
bijvoorbeeld de anti-Oranje propaganda - had enkel behoefte aan een duidelijk doel. Farnese boekte zijn communicatieve
successen voornamelijk met dit anti-Oranje beeld. Monica Steunlands verklaard dat met het verlies van de Spaanse Armada
het imago van Farnese brak, wat als gevolg had dat . Wat hier echter ontbreekt is dat met het plotseling vijand maken
van de Engelse, wat overigens tegen het advies van Farnese in gebeurde, het volk niet langer een helder beeld heeft bij
het doel van het Regime. Daarmee had het Regime de handen vol aan de oorlog met de Fransen. Is het daarmee niet zo dat
de communicatie van Farnese uitbleef na het verlies van de Armada voortkomt uit de complexe situatie waarin het Regime
zich bevind, een complexe situatie welke niet duidelijke naar het volk is te communiceren. Een situatie waar Farnese
het zelf niet mee eens was.

Wat Monica Steunlands beweerd is dat de Habsburgse communicatie afhangt van de individuele intiatieven van de landvoogd, echter ontbreekt aan deze bewering de ondersteuning van het Habsburgse regime.
Granvelle was een van de voornaamste actoren binnen het ontstaan van de Habsburgse regime (quote uit boek) een daarbij wordt er blijkbaar ook door verschillende actoren buiten het Regime propaganda gemaakt.
Dit is verder terug te verhalen aan het verlies van Farnese's communicatie, aangezien Farnese in Spanje en Italie een slecht imago had gekregen omdat zijn macht aanzienlijk was gegroeid, daarmee voorkomend
uit de hoefte om Farnese te onderdrukken. Kan het daarmee niet zo zijn dat de ondersteuning welke Farnese kreeg weg was gevallen?

Farnese boekte successen in Frankrijk echter kwamen deze niet aan het licht, deze kwamen pas in omloop na de dood van Farnese.


\newpage

\section{Het autoritair karakter van het zuiden}
\textit{Schets de situatie van het zuiden en de komst van Alexander Farnese. Wat leidde ertoe dat Farnese ervoor koos het
     publiek te gebruiken en waarom was de communicatie niet langer autoritair, zoals dat van Alva. Vergelijk deze twee
     landvoogden en de situatie waar ze zich in bevonden. Ik negeer Don Juan aangezien hij geen grote invloeden heeft
     uitgeoefend.}

\section{De propaganda van Alexander Farnese}
\textit{Beschrijf de inhoud propaganda van Alexander Farnes, gebruik misschien "Not as bad as all that: The strategies and effectiveness of Loyalist propaganda in the early years of Alexander Farnese's governorship"}

\section{De discussie cultuur van het noorden}
\textit{Schets de situatie van het noorden, beschrijf de discussie maatschappij en plaats hierbinnen de propaganda als een
     instrument om politieke acties uit te voeren. Beschrijf te zwakheden van de staat en hoe dit leidde tot een nationaliteit of identiteit van de Nederland. Ontleed hieruit dat propaganda noodzakelijk was om een identiteit te creeeren tussen de verschillende provincies wie geen verleden met elkaar deelde. }

\section{Conclusie}
\textit{Maak een samengevatte vergelijking welke aspecten zijn er hetzelfde? religie standpunten van de staat (romans 13), de zwakheden, }

\newpage\printbibliography{}

\end{document}

%------------------------------------------------------------------------------
% End of journal.tex
%------------------------------------------------------------------------------
