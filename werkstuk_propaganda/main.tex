\documentclass[12pt]{amsart}

\usepackage{times}
\usepackage[utf8]{inputenc}
\usepackage{amsmath, amsthm, amscd, amsfonts, amssymb, graphicx, color}
\usepackage[bookmarksnumbered, colorlinks, plainpages=false]{hyperref}
\usepackage{bookmark}
\usepackage{csquotes}

\usepackage[dutch]{babel}

\usepackage[backend=biber]{biblatex}
\addbibresource{zotero.bib}

\hypersetup{
     colorlinks   = true,
     citecolor	  = blue
}

\textheight 22.5truecm \textwidth 14.5truecm
\setlength{\oddsidemargin}{0.35in}\setlength{\evensidemargin}{0.35in}

\setlength{\topmargin}{-.5cm}

\newtheorem{theorem}{Theorem}[section]
\newtheorem{lemma}[theorem]{Lemma}
\newtheorem{proposition}[theorem]{Proposition}
\newtheorem{corollary}[theorem]{Corollary}
\theoremstyle{definition}
\newtheorem{definition}[theorem]{Definition}
\newtheorem{example}[theorem]{Example}
\newtheorem{exercise}[theorem]{Exercise}
\newtheorem{conclusion}[theorem]{Conclusion}
\newtheorem{conjecture}[theorem]{Conjecture}
\newtheorem{criterion}[theorem]{Criterion}
\newtheorem{summary}[theorem]{Summary}
\newtheorem{axiom}[theorem]{Axiom}
\newtheorem{problem}[theorem]{Problem}
\theoremstyle{remark}
\newtheorem{remark}[theorem]{Remark}
\numberwithin{equation}{section}

\begin{document}
\setcounter{page}{1}

\noindent {\small 2324-S1 Themacollege I WGR 103}\hfill     {\small \today} \\
% title, issn/data
{\small Werkstuk over propaganda tijdens de Nederlandse Opstand.}\hfill
{\small } %volume, url

\centerline{}

\centerline{}

\title[Short Title]{De communicatieve gebruiken van het Habsburgse regime in
     verhouding met de propaganda van de rebellen - tijdens de Opstand,
     1568\ldots1648}

\author[J. Vermeltfoort]{Vermeltfoort, Jenny$^1$}

\address{$^{1}$ 3787494, Student Faculteit Geesteswetenschappen, Leiden
     Universiteit, Leiden, Nederland.}
\email{\textcolor[rgb]{0.00,0.00,0.84}{j.vermeltfoort@umail.leidenuniv.nl}}

%\dedicatory{This paper is dedicated to Professor ABCD}
%\subjclass[2020]{Primary 46L55; Secondary 44B20.}
%\keywords{Analysis, PDEs, algebra, number theory, applied mathematics}
%\date{Received: xxxxxx; Revised: yyyyyy; Accepted: zzzzzz.
%\newline \indent $^{*}$ Corresponding author}

\begin{abstract}
\end{abstract} \maketitle

\section{Introductie en toelichting}

% Hoe manifesteerde de propaganda van het noorden
% Hoe manifesteerde de communicatie van het regime, wat was de reactie op het noorden, 
% 

% de verhouding op het gebied van sociale verschillen tussen het noorden en zuid, denk aan religie, zwakheden van het noorden, gebruik Jasper van de Steen. Propaganda werd een noodzaak, mogelijk een gevolg van de publieke arena, bookshop van de wereld (Andrew Pettegree)

% De reactie en effect van het regime op de propaganda, en hun ondermijning van de nederlandse dissaffectie, Paul Regan, 
% discern dismay about abla's violence, Gustaaf Janssens
% local officals obstructed laws against heresy, Juliaan Wolter's

% kort vertellen over de communcatie methodes van het regime waren aan de hand van monica stenlands, tradionele communcatie, romans 13, etc
% vertellen over de latere propaganda van het regime welke met Don Jou, Farnesse en aartsbishoppen toch manifesteerde, monica stenlands

% Dit artikel zal omschrijven wat was de verhouding tussen de Habsburgse communicatie en de politieke propaganda van de Rebellen?

% Zoals eerder beschreven is de reatie van het regime ingeperkt gebleven en zorgde het niet voor het verwachte effect, vandaar de vraag waarom er vast werd gehouden aan de tradionele methoden en waarom later de propaganda toch manifesteerde. Omschrijven hoe de latere propaganda van het regime veranderd was vergelekene met dat van de originele communicatie methoes, onderdrukking. Waar kwam dit uit voort? Leerde ze van het noorden?

% wat was het effect van de habsburgse propaganda op de noordelijke propaganda?

% conclusie: het noorden maakten sterk gebruik van propaganda tijdens de. De propaganda kwam tot bloei door de verdrukking van het zuiden en de intellectuele infrastructuur in het noorden. Het zuiden onderschatte de ontevredenheid van het volk tijdens de destructieve periode van Alba. 

% religieus verschil
% infrastructuur verschil
% situatie verandering na alva.

\noindent Antwerpen, 1584, met een veertien maanden durig beleg kwam de stad in handen van de barmhartige en
strategische bekwame landvoogd, Alexander Farnese. Voorafgaand het bewind van deze Habsburgse landvoogd waren de
generaliteitslanden in een wanordelijke staat, er bestond een agitatie tussen de provincies van de generaliteits landen
en met de opkomende godsdienstvrijheid liep het conflict tussen de katholieke en calvinisten hoog op. Er heerste
onvrede tegenover het Habsburgse regime na het hardvochtige bewind van de Hertog van Alva en de achterblijvende
communicatie methodieken van Don Jou. Farnese speelde strategisch in op de zwakte van de generaliteitslanden met een
aantal succesvolle herrovering van verschillende steden, en doormiddel van pro-regime propaganda, wat uniek was voor
het regime, wist hij het imago van het Habsburgse regime te herstellen. Calvinsten binnen de herroverde steden werden
niet langer vervolgd, de barmhartige Farnese besloot de calvinisten de ruimte te geven vredevol het land te verlaten.
Hij wist effectief betrokkenheid aan te gaan met het volk door in te spelen op de heersende grieven. Dit verstekte niet
enkel het imago van het regime, maar ook dat van hemzelf. Toch sloeg de landvoogd er niet in de tradionele communicatie
methodieken van het regime volledig te verwerpen en wist zijn technieken niet te institutionaliseren.
Na de verwoesting van de Spaanse Armade in 1588 werd de reputatie van de landvoogd geschaad, wat ertoe leidde dat de
communicatie wegviel, wat met verschillende andere factoren ertoe leidde dat de generatliteitslanden verloren gingen.

De manier waarop Farnese de weg door publieke relaties trachtte laat zien wat de kracht is van een succesvolle
propaganda campagne. Dit word gereflecteerd in het boek van Monica Stensland, waar Monica de communicatie methodieken
van de verschillende Habsburgse landvoogde beschrijft. Ze laat zien dat de traditionele publieke relatie van het Regime
voornamelijk een autoritair karakter beslaat en dat later de communicatie ontwikkelde tot succesvolle propaganda.

In het noorden van het de Lage Landen had de publieke arena een uitzonderlijk effect op de identiteit van de
Nederlanden, beschreven door Judith Pollman, en Andrew Spicer. In de bundel word doormiddel van een aantal artikelen de
publieke arena en het effect van de noordelijke propaganda geschetst. Andrew Sawyer en zijn artikel \textit{Medium and
     Message} laat het karakter zien van de noordelijke propaganda, waar vrijheid en gerechtigheid een hoofdrol
spelen, met een aversie naar monarische macht en waar er werd gepleit voor een olichargische machts structuur.	Henk
van Nierop laat zien dat geruchten net zo belangrijk zijn als de pamfletten en het artikel \textit{A Provincial News
     Community in Sixteenth-Century Europe} van Andrew Pettegree beschrijft het belang van de locale nieuws
gemeenschappen. De artikelen binnen deze bundel laten het belang zien van de publieke arena in de toekomst en het
vormen van de Lage Landen.

Jasper van der Steen, met artikel \textit{Remembering the Revolt of the Low
     Countries Historical Canon Formation in the Dutch Republic and Habsburg Netherlands}, merkt op dat de het
nationaal karakter van het noorden onstaat door de zwakte van de staat, religieus verdeeld en ontbrekende van een sterk
centraal gezag.

Het is opmerkelijk dat in tijden van zwakheden van de rebellen en het Habsburgse Regime tijdens de opstand leidde tot
het gebruik van propaganda - de rebellen, wanneer zij op het punt stonden privileges te verliezen en tijdens de
Tachtigjarige Oorlog werd Hertog van Alva als vijandbeeld spoedig gebruikt voor uniformiteit - het Habsburgse regime
wanneer zij de generaliteitslanden verloren. Kan er door een link te leggen tussen de communicatie methodieken van het
Habsburgse Regime en de Republiek een karakteristiek gekoppeld worden aan de 16de eeuwse propaganda?
Was het zo dat de propaganda een facade was voor de zwakheden van de staten en is de opkomst van propaganda
een natuurlijke reactie op de situatie - waar in het Habsburgse Regime propaganda enkel gekoppeld wordt aan Farnese
door Monica Stensland - in de Republiek het gevolg van de discussie maatschappij, waar invloed enkel beoefend kan
worden binnen de publieke arena, zoals beschreven door Judith Pollman - of beinvloedde de zwakheden van de staat welke
Jasper van der Steen beschrijft? Dit artikel zal een overzicht scheppen op het ontstaan van propaganda, haar effecten
op het imago van de staat beschrijven, en vergelijking opstellen tussen de propaganda van de Habsburugse Regime en dat
van de Republiek.

\printbibliography{}

\end{document}

%------------------------------------------------------------------------------
% End of journal.tex
%------------------------------------------------------------------------------
