\documentclass[11pt]{amsart}

\usepackage{tgpagella}
\fontfamily{qpl}
\usepackage[utf8]{inputenc}
\usepackage{amsmath, amsthm, amscd, amsfonts, amssymb, graphicx, color}
\usepackage[bookmarksnumbered, colorlinks, plainpages=false]{hyperref}
\usepackage{bookmark}
\usepackage{csquotes}
\usepackage{dirtytalk}
\usepackage[dutch]{babel}
\usepackage[norule,bottom]{footmisc}
\usepackage[style=verbose-ibid,backend=biber]{biblatex}
\addbibresource{zotero.bib}

\hypersetup{
     colorlinks = true,
     citecolor = blue,
     urlcolor = blue,
     linkcolor = blue,
}

\textheight 22.5truecm \textwidth 14.5truecm
\setlength{\oddsidemargin}{0.35in}\setlength{\evensidemargin}{0.35in}

\setlength{\topmargin}{-.5cm}

\begin{document}
\setcounter{page}{1}

\noindent {\small 2324-S1 Werkcollege Nederlandse Geschiedenis}\hfill     {\small \today} \\
% title, issn/data
{\small College 5, vragen bij het handboek.}\hfill
{\small } %volume, url

\centerline{}

\centerline{}

\title[Bataafse Revolutie]{Revolutie. Patriotten, Bataven en Fransen.}

\author[J. Vermeltfoort]{Vermeltfoort, Jenny$^1$}

\address{$^{1}$ 3787494, Faculteit Geesteswetenschappen, Leiden
     Universiteit, Leiden, Nederland.}
\email{\textcolor[rgb]{0.00,0.00,0.84}{j.vermeltfoort@umail.leidenuniv.nl}}

%\dedicatory{This paper is dedicated to Professor ABCD}
%\subjclass[2020]{Primary 46L55; Secondary 44B20.}
%\keywords{Analysis, PDEs, algebra, number theory, applied mathematics}
%\date{Received: xxxxxx; Revised: yyyyyy; Accepted: zzzzzz.
%\newline \indent $^{*}$ Corresponding author}


\maketitle

\section*{Opdracht 1}
\label{opdracht1}
Volgens Friso Wielenga is kern van de patriotten beweging het herstel van de vroege kracht en het verlangen naar medezeggenschap in de lokale besturen. \autocite[178]{wielengaGeschiedenisVanNederland2022} Voor het eerst in de Republiek ging het hier niet langer om een factiestrijd, maar om een ideologisch conflict tussen verschillende partijen. \autocite[179]{wielengaGeschiedenisVanNederland2022} De patriotten zouden progressief genoemd kunnen worden, versus de orangisten, gekarakteriseerd als conservatief. De beweging werd overigens actief beoefend door de middenklasse, en er onstond \say{een sociale bovenlaag die neerkeek op het koopmandschap}. \autocite{wielengaGeschiedenisVanNederland2022}

Binnen het historisch debat ligt de vraag of het hier gaat om een moderne beweging. Wielenga beantwoord dit met twee standpunten, aan de ene kant werd er door de patriotten een traditionele standhouding ge"eist. \autocite[182]{wielengaGeschiedenisVanNederland2022} Een stadhouder waar de macht niet langer onafhankelijk was, maar waar verloren privileges en rechten terugkeerde. \autocite[182]{wielengaGeschiedenisVanNederland2022} 

Aan de andere kant beschrijft Wielenga dat het verlangen van medezeggenschap binnen de lokale politiek steun geeft aan de moderniteit van de beweging. \autocite[182]{wielengaGeschiedenisVanNederland2022} 
Deze eis staat in direct conflict met de traditionele status quo.
Mogelijk waren de patriotten zich er niet van bewust wat medezeggenschap inhoudelijk inhield en wat voor impact dit zou hebben op de bestuurlijke structuren. Aangezien het verschil tussen het traditionele en het vereisten dermate groot is zou er dus verondersteld kunnen worden dat het hier gaat om een moderne beweging

\section*{Opdracht 2}
De \textit{penalties of the pioneer} wordt door ecnomische historici gebruikt voor de "straf" die betaald werd door degene wie als eerst een technologische vooruitgang bewerkstelligd. De economie van de Republiek  was na de zeventiende eeuw gebaseerd was op wat in de achtiende eeuw als onderontwikkelde technologi"en werden bevonden. Dit betekende dat concurrentie, met de nieuwste technologi"en, sterker in de productieve schoenen stonden. Investeringen in nieuwe technologi"en door de Republiek was niet rendabel en daardoor werd innovatie minder interessant.\autocite[187]{wielengaGeschiedenisVanNederland2022}
Een modern voorbeeld hiervan is China met betrekking tot technologische advances. Het is een natuurlijke aanname dat het goedkoper en sneller is om innovatie toe te passen op gekopieerde technologie. Dit reflecteerd in de technologische advances welke China ondervind. Vergeleken met Europa en de US, is China met haar nieuwe kopieer beleid aanzienlijk vooruit gelopen en zelfs een tech gigant geworden.

\section*{Opdracht 3}
Wat Maarten Prak beschrijft is dat de Republiek opereerde met een corporatief systeem. Lokale, Regionale, en gewestelijke belangengroepen beschikten over een grote mate van zelfstandigheid en opereerden in ingewikkeld krachtenveld.\autocite[188]{wielengaGeschiedenisVanNederland2022}

In de zeventiende eeuw zorgde het Republiekse decentrale karakter voor vrijheid van individuele corporaties om economische innovaties te verwezelijken. Eigenbelang leidde tot financieele hoogte punten, waar de rest van het volk van mee kon teren. Hoewel deze corporaties redelijk autonoom waren, besond er een afhankelijkheid gekoppeld aan de algemene staat van de Republiek en haar buitelandse polieke handelen. De demografie had direct invloed op de economische toestand van deze corporaties. Een lagere populatie leidde tot minder arbeidskrachten, een oorlog met Engeland zorgde voor handel stagnatie, een oorlog met Frankrijk leidde tot hoge belastingen. Door deze afhankelijkheid, en daarmee het ontbreken van centrale regelgeving en bestuurlijk uitvoeren, onbrak er institutionele flexibiliteit. 

In de zeventiende eeuw was de situatie geschikt voor economische hoogstanden, met de migraties vanuit het zuiden vanwege de Opstand (arbeidskracht), de wereldse monopolie in het Oost-indische gebied (handel), de conflicten tussen verschillende Europese landen(verhoogde export). Dit leidde allemaal tot een sterke economische staat. Deze fragiele en onstabiele situatie, een economie gebaseerd op koststondige eigenschappen, kwam ter val in de achtiende eeuw, na het verlopen van deze eigenschappen. De interne en externe politieke en maatschappij situatie werd niet door een centraal gezag verzorgd, ook werd er niet gestuurd op vernieuwing of opleggen van nieuwe fundamenten.

\section*{Opdracht 4}
Kunnen we spreken over een Nederlandse revolutie aan het eind van de achttiende eeuw? Aan het eind van de achttiende eeuw waren de Nederlanden radicaal genationaliseerd. Met de verschillende centralisatie structuren voortkomende uit de Franse bezetting, zoals de nieuwe grondwet, het Burgelijk Wetboek, het Weboek van Strafrecht, en zelfs een nationale feestdag. Het burgerschap was niet langer begrenst tot een specifieke groep mensen, elke inwoner kon nu burgelijke vrijheden genieten. Vrijheid van meningsuiting, vergadering en godsdienst waren gegarandeerd.

Ik benoem hier bewust niet de Nederlandse patriotten beweging. De verlangens van de patriotten, beschreven in \hyperref[opdracht1]{Opdracht 1}, kwamen niet tot een concreet en langdurige politieke hervorming.

Puur uitgaande van de radicale verschillen op het gebied van nationaliteit en individuele vrijheid, tijdens periode van de bataafse revolutie en de Fransen bezetting, zou ik veronderstellen dat er zeker sprake is van een Nederlandse revolutie.

\section*{Opdracht 5}
De federalisten hielden vast aan de oude federale structuren, hier tegenover stonden de unitarissen wie radicale inspiratie vonden in de Franse revolutie.\autocite[200]{wielengaGeschiedenisVanNederland2022} Ze wilde een eenheidsstaat en een einde aan de corporatistische grondslag.\autocite[199]{wielengaGeschiedenisVanNederland2022}

Uiteindelijk kwamen de unitarissen aan de macht met een coup, gesteund door de Fransen.\autocite[200]{wielengaGeschiedenisVanNederland2022} Het resultaat was een radicale unitaristische Staatsregeling waar onder andere de provincies werden afgeschaft.\autocite[200]{wielengaGeschiedenisVanNederland2022} Agentschappen werden ingevoerd en centralisatie \say{op het gebied van onderwijs, armenzorg, waterstaat en economie.}\autocite[200]{wielengaGeschiedenisVanNederland2022}





\printbibliography{}

\end{document}

%------------------------------------------------------------------------------
% End of journal.tex
%------------------------------------------------------------------------------
