\documentclass[10pt]{amsart}

\usepackage{tgpagella}
\fontfamily{qpl}
\usepackage[utf8]{inputenc}
\usepackage{amsmath, amsthm, amscd, amsfonts, amssymb, graphicx, color}
\usepackage[bookmarksnumbered, colorlinks, plainpages=false]{hyperref}
\usepackage{bookmark}
\usepackage{csquotes}
\usepackage{dirtytalk}
\usepackage{float}

\usepackage[dutch]{babel}

\usepackage[style=verbose,backend=biber]{biblatex}
\addbibresource{zotero.bib}
\DeclareBibliographyAlias{artwork}{misc}

\hypersetup{
     colorlinks = true,
     citecolor = blue,
     urlcolor = blue,
     linkcolor = blue,
}

\textheight 22.5truecm \textwidth 14.5truecm
\setlength{\oddsidemargin}{0.35in}\setlength{\evensidemargin}{0.35in}

\setlength{\topmargin}{-.5cm}

\begin{document}
\setcounter{page}{1}

\noindent {\small 2324-S1 werkcollege Oude Geschiedenis}\hfill     {\small \today} \\
% title, issn/data
{\small Opdracht 3: Polythe"isme.}\hfill
{\small } %volume, url

\centerline{}

\centerline{}

\title[Polythe"isme]{Polythe"isme}

\author[J. Vermeltfoort]{Vermeltfoort, Jenny$^1$}

\address{$^{1}$ 3787494, Faculteit Geesteswetenschappen, Leiden
     Universiteit, Leiden, Nederland.}
\email{\textcolor[rgb]{0.00,0.00,0.84}{j.vermeltfoort@umail.leidenuniv.nl}}

\maketitle

\section*{Opdracht 1}

\noindent Wat de tekst beschrijft is dat een god type macht representerend. Een god binnen het pantheon wordt niet als een geïsoleerd onderwerp gezien, maar gezien als zijn positie binnen het systeem van krachten. Daarmee reflecteert de godenwereld een limiet aan deze krachten en hun hiërarchische evenwicht. Hierdoor worden categorieën als almacht, alwetendheid en oneindige macht uitgesloten. \autocite{vernantGreceAncienneEtude1976}

Of dit een bewijskrachtig argument is, is niet geheel zeker. De tekst verklaard dat aangezien de Grieken over het algemeen als intellectueel consistent worden gezien, het leven in een soort religieuze disorde onwaarschijnlijk is. Het argument wordt dus gebaseerd op de aanname dat het intellectueel vermogen van de Grieken direct invloed op het religieus denken. Het zou kunnen zijn dat het religieus beeld niet direct werd verklaard door Grieks intellectuele consistentie, maar dat de goden gezien werden als een onverklaarbare hogere macht. Het zou kunnen dat de goden door verschillende verspreidde groepen werd geinterpreteerd. 

Verder worden er in dit stukje tekst geen referenties gebruikt, hierdoor zou je kunnen veronderstellen dat wat er gevraagd wordt niet bedoeld is als een bewijskrachtig argument. Het lijkt erop dat de schrijver een retorische vraag steld om de lezer aan het denken te zetten.


\section*{Opdracht 2}\label{opdracht2}
Wat beide teksten\autocite{parkerPolytheismSocietyAthens2005}\autocite{burkertGreekReligionArchaic1985} verklaren is dat het goddelijk stelsel van de Grieken geen systematische structuur bevat zoals de tekst van Pierre Vernant \autocite{vernantGreceAncienneEtude1976} dat oppert. 

Burkert beschrijft dat de persoonlijkheid van een god door vier factoren werd samengesteld. Ten eerste de cultus gelokaliseerd in tijd en plaats met haar eigen rituele programma, ten tweede de heilige naam, ten derde de mythes die er werden verteld, ten vierde het beeld wat de god gepresenteerd.\autocite{burkertGreekReligionArchaic1985}

Dit komt overheen met het beeld wat Parker beschrijft waar het lokale pantheon geen product is van systematische logica, maar voortkomt uit lokale interpretaties van ervaringen. Een voorbeeld hiervan is de reflectie van Demeter in een graan gewas en Dionysus in een wijnrank. Aangezien er twee goden worden gepresenteerd voor de teelt van gewassen, welke god moet er dan vereerd worden door de boer? Dit laat het lokale en persoonlijke karakter van de interpertatie van de goden zien en hoe er verschillende attributen aan de goden worden gekoppeld. Daarmee is het Griekse pantheon een map van ervaringen, waarbij de map een conceptuele schetsing is van het goddelijk stelsel.\autocite{parkerPolytheismSocietyAthens2005}

Het \textit{Theogeny of Hesoid} wat gebruikt word door Parker\autocite{parkerPolytheismSocietyAthens2005} is een structurele beschrijving van de goden geschreven door de griekse schrijver Hesoid\autocite{atsmaHESIODTHEOGONY2017}. Hier worden de origines en stambomen van de goden beschreven. De tekst van Parker ontkracht echter dat Hesoid de systematische logica een directe reflectie is van de Griekse godenwereld. Zoals hierboven beschreven is de godenwereld eerder een chaotische opvatting van verschillende lokale en persoonlijke bevindingen.

\section*{Opdracht 3}
Met de hymne over de god Isis\autocite{i.AretologieVanIsis} kunnen een aantal termen gekoppeld worden aan de godin. Hier volgende; almachtig, absoluut, gerechtvaardigd, alwetend, \textit{creator diety}.

\section*{Opdracht 4}
Het lijkt er in hymne op dat de god Isis de god van alle goden is, aangezien vrijwel alles wat de mens ondervind een uitkomst is van haar ontwikkelingen.
Maar kijked naar de tekst word Isis wel geplaatst in de stammenboom van verschillende andere goden. Hierdoor kan er strikt genomen geconcludeerd worden dat het hier niet gaat om monothe"isme, maar om polythe"isme.

\section*{Opdracht 5}
De definitie welke Vernant\autocite{vernantGreceAncienneEtude1976} steld conflicteerd met het almachtige karakter van Isis wat in de hymne\autocite{i.AretologieVanIsis} aan bod komt.

De tekst van Vernant beschrijft dat de individuele goden een strikte gelimiteerde positie bekleden binnen het systeem van machten. Oftwel een god is uitsluitend alleen verantwoordelijk binnen haar gekarakteriseerd kader. Hierdoor is het niet mogelijk dat er een god bestaat wie mede verantwoordelijk is voor de kaders van de verschillende andere goden, deze posities zijn namelijk al bekleed. Daarmee, in de ogen van Vernant, is er voor de god Isis, gekenmerkt als almachtig en alwetend, geen positie om te bekleden. 

\section*{Opdracht 6}

De hymne\autocite{vanderlipHymneVoorIsis1972} beschouwdt dat Isis de moeder van alle goden is. Isis is daarmee de gecombineerde representatie van alle goden.

Apuleius\autocite{apuleiusMetamorphoses02eeeuwn.Chr.} beschrijft Isis als de belichaming is van elke god binnen het pantheon. Apuleius beschrijft de goden wereld niet langer als polythe"istisch, maar als monothei"stisch. Met zijn zin "in welke gedaante het geoorloofd is U aan te roepen" drukt hij uit dat Isis de belichaming is van elke individuele god binnen het pantheon. Oftewel elke individuele god is een gedeeltelijke reflectie van de god Isis.

\section*{Opdracht 7}
Wanneer deze twee teksten worden vergeleken met de ongestructureerde lokale goddenwereld gedefinieerd in de teksten van Parker en Boswell, zie \hyperref[Opdracht 2]{opdracht2}, is het lastig om de karakteristieken van deze lokale goden of ervaringen te plaatsen binnen een gecentraliseerd kader. Hierdoor wordt het concept van de god Isis opzichzelf al een mengeling van eigenschappen, resulterende in verschillende versies van Isis. De verering van Isis wordt nu gedaan met verscheidene motieven, waardoor de godenwereld niet langer polythe"istisch is, maar duid op totstandkoming van monothe"isme.

\section*{Opdracht 8}
Er wordt met Paulus's\autocite{paulusHandelingenApostelen1996} god een afbraak gemaakt op de polytheïstische wereld van het pantheon. In plaats van het verbinden of samenvoegen van verschillende interpretaties van goden zegt hij concreet dat er maar een enkele god bestaat. Een god verantwoordelijk voor het volledig menselijk bestaan. Hierdoor ontbind hij de theologische opvattingen van de menselijke ervaringen en plaatst de god in een kader van transcendentie.

Enige samenhang met de god Isis lijkt er echter wel te zijn. Net als Isis heeft ook Paulus's god de verantwoording over het volledige onstaan van de mensheid, waar ook Paulus's god een alles omvattende representatie is van wat de mensheid ervaard.

Verder lijkt Paulus's schrift een conflicterend beeld te bevatten van god wanneer hier metafysica word toegepast. Hij zegt namelijk: \say{Ook laat Hij zich niet door mensenhanden bedienen, alsof Hij iets nodig had, want zelf geeft Hij aan allen leven en adem en alles.}\autocite{paulusHandelingenApostelen1996}. Hier lijkt het alsof Paulus god plaatst in een onvatbaar hogere macht, waardoor hij god ontbind van menselijk handelen. Echter zegt hij vervolgens later in zijn tekst: \say{God zouden zoeken en Hem wellicht tastenderwijs zouden vinden; Hij is immers niet ver van ieder van ons. (...) Want door Hem leven wij, bewegen wij en zijn wij, zoals ook enkele van uw dichters hebben gezegd: Wij zijn van goddelijke afkomst.}\autocite{paulusHandelingenApostelen1996}. \say{Tastenderwijs} en \say{godelijke afkomst} lijken te duiden op de inherentie van god binnen het menselijk bestaan. Hier staan transcendentie en immanentie niet langer tegenover elkaar maar lijkt er een mengeling van de twee metafysische begrippen te onstaan.

\section*{Opdracht 9}
Het is moeilijk in overeenstemming te komen om een altaar voor een onbekende god te plaatsten binnen de Griekse godenwereld. Vernant verklaard namelijk dat elke ervaring of elke macht binnen het systeem van machten een positie bekleed.\autocite{vernantGreceAncienneEtude1976} Een onbekende god past niet binnen dit systeem van machten, aangezien een onbekende positie automatisch door de werking van het systeem een concrete positie krijgt toegeschreven. Een onbekende positie bestaat niet aangezien er geen ervaringen gekoppeld kunnen worden een aan onbekend iets.

\section*{Opdracht 10}
\label{opdracht10}
Uit teksten blijkt dat de Grieken in een onzekere religieuze wereld het geloof beoefende. Voor elke ervaring of macht werd er een specifieke god verantwoordelijk gesteld. Hierdoor werd het kader van deze goden betrekkelijk klein. Het was noodzakelijk om een groot scala aan goden aan te spreken om een bepaalde paraplu behoefte te opperen, oftewel een behoefte welke een gedeeltelijke betrekking heeft op de verantwoordelijkheid van verschillende goden. Het was ook niet altijd duidelijk welke goden er exact aangesproken moesten worden, dit mede door het complexe systeem van verantwoordelijkheden binnen het Griekse goden netwerk. 

\section*{Opdracht 11}
\label{opdracht11}
Zie \hyperref[Opdracht 10]{opdracht10}, hierop volgend bestaat er in het antieke geloof nog meer onzekerheid. Het was voor de gelovige onduidelijk of een god gemanifesteerd kon zijn in een menselijk lichaam. Dit blijkt uit de tekst van de schipbreukeling, onzeker of de persoon waar tegen gesproken werd een god of een mens was. Ook in de passage uit het Nieuwe Testament worden goden gepresenteerd als menselijke gedaantes, \say{De goden zijn mensen geworden en naar ons afgedaald!}. In de voettekst worden Nympen genoemd, een halfgodin vaak verbonden met een bepaalde plek of plantensoort, daarbij een god belichaamd in een menselijk lichaam. Het argrarische handboek beschrijft onzekerheid over aan welke god er offers gedaan moeten worden, dat er een offer gedaan moet worden is wel zeker, wat in lijn is met de complexiteit van het Griekse goddelijk systeem.

\section*{Opdracht 12}
De problemen beschreven in \hyperref[Opdracht 11]{opdracht11} zijn niet enkel problemen van het polythe"isme. In het monothe"isme is zijn er ook degelijk dit soort problemen te vinden. Zo is het chistendom sterk gebaseerd op de menselijke incarnatie van God, Jesus. Gezien als de zoon van god, gestuurd door god, de capaciteit om wonderen te verichten net als Paulus.


\section*{Opdracht 13}   
Nock verklaard dat Met judaïsme en christendom kwam de adhesie van de wil aan een theologie, oftewel het geloof.\autocite{nockConversionOldNew1933} De heidenen legde machten op aan een god door ervaringen of materie te manifesteren tot een god. Bij religie was dit omgekeerd, god legde macht op aan de mens. De gelovige moesten hun levenswijze onderwerpen aan de door god voorgeschreven normen en waarden. 

\section*{Opdracht 14}
Er wordt door Nock\autocite{nockConversionOldNew1933} een onderscheid gemaakt tussen het jodendom en christendom welke worden gezien als religies, de overigen geloven waren een vorm van cultussen. In deze cultussen werd er niet meer verwacht dan de aanbidding van het bovennatuurlijke. De cult was niet afhankelijk van de principes en waardes, het was dus geen levenswijze, zoals een religie dat heeft, dit werd ingevuld met filosofie. Hierdoor kan bekering niet mogelijk zijn binnen het polythe"isme, aangezien er in de \say{ziel}, ofwel de levenswijze, van een mens geen hervorming plaatsvind.


\section*{Opdracht 15}






\newpage \printbibliography{}

\end{document} 

%------------------------------------------------------------------------------
% End of journal.tex
%------------------------------------------------------------------------------
